% Appendix D: Glossary of Terms
%
% This appendix provides definitions for key terms used throughout
% the dissertation.

\chapter{Glossary}
\label{app:glossary}

This glossary provides definitions for key terms used throughout the
dissertation. Terms are organized alphabetically within thematic groups.

\section{Confidence and Uncertainty}

\begin{description}[style=nextline,leftmargin=2cm]

\item[Confidence]
A value in $[0,1]$ representing epistemic commitment to a belief.
Unlike probability, confidence does not require normalization across
complementary beliefs. A value of 0.5 represents maximal uncertainty,
not a 50\% probability.

\item[Confidence Algebra]
The algebraic structure of confidence operations. CLAIR uses three
monoids: multiplication $([0,1], \times, 1)$ for derivation, minimum
$([0,1], \min, 1)$ for conservative combination, and probabilistic OR
$([0,1], \oplus, 0)$ for aggregation.

\item[Probabilistic OR ($\oplus$)]
The operation $a \oplus b = a + b - ab$ that aggregates independent
evidence. Also known as the algebraic sum t-conorm.

\item[T-norm]
A binary operation on $[0,1]$ modeling conjunction in fuzzy logic.
CLAIR's multiplication and minimum are t-norms.

\item[T-conorm]
A binary operation on $[0,1]$ modeling disjunction in fuzzy logic.
CLAIR's $\oplus$ is a t-conorm.

\end{description}

\section{Belief and Justification}

\begin{description}[style=nextline,leftmargin=2cm]

\item[Belief]
A typed value in CLAIR carrying epistemic metadata: value, confidence,
provenance, justification, and invalidation conditions.
Notation: $\mathsf{Belief}\langle\tau\rangle$.

\item[Belief Revision]
The process of updating beliefs in response to new information.
CLAIR extends AGM theory to handle graded DAG-structured beliefs
with justification-based retraction.

\item[DAG (Directed Acyclic Graph)]
A graph where edges have direction and no cycles exist. CLAIR uses
DAGs for justification structure to ensure well-foundedness.

\item[Defeat]
When evidence reduces the confidence of a belief. CLAIR distinguishes
\emph{undercutting} defeat (attacks the inference) from \emph{rebutting}
defeat (provides counter-evidence).

\item[Justification]
The epistemic support for a belief, represented as a DAG with labeled
edges (support, undercut, rebut). Well-foundedness requires acyclicity.

\item[Provenance]
Metadata tracking where a belief came from: literal, input, derived,
training, or oracle.

\item[Rebut]
A type of defeat that attacks the conclusion directly by providing
competing evidence. Computed as $\mathsf{rebut}(c_{\mathit{for}}, c_{\mathit{against}})
= c_{\mathit{for}} / (c_{\mathit{for}} + c_{\mathit{against}})$.

\item[Reinstatement]
When a defeated belief regains confidence because its defeater is
itself defeated. In CLAIR, reinstatement emerges compositionally
from bottom-up evaluation.

\item[Undercut]
A type of defeat that attacks the inferential link rather than the
conclusion. Computed as $\mathsf{undercut}(c, d) = c \times (1-d)$.

\end{description}

\section{Self-Reference and Provability}

\begin{description}[style=nextline,leftmargin=2cm]

\item[Anti-Bootstrapping]
The principle that self-soundness claims cannot increase confidence.
If a system claims ``my beliefs are sound,'' this claim cannot justify
higher confidence than the original beliefs.

\item[CPL (Confidence-Bounded Provability Logic)]
A novel extension of G\"{o}del-L\"{o}b provability logic with graded
modalities. The graded L\"{o}b axiom includes a discount function
$g(c) = c^2$.

\item[CPL-finite]
A decidable fragment of CPL using a finite confidence lattice such as
$L_5 = \{0, 0.25, 0.5, 0.75, 1\}$.

\item[GL (G\"{o}del-L\"{o}b Logic)]
The modal logic of provability, characterized by the L\"{o}b axiom
$\Box(\Box A \to A) \to \Box A$. CLAIR's belief logic resembles GL.

\item[Graded Modality]
A modal operator parameterized by a degree, such as $\Box_c$ meaning
``believed with confidence at least $c$.''

\item[L\"{o}b's Theorem]
A theorem in provability logic stating that if a system can prove
$\Box P \to P$ for arbitrary $P$, it can prove $P$. This limits
self-soundness claims.

\item[Stratification]
Organizing beliefs into levels where level-$n$ beliefs can only
reference level-$(n-1)$ beliefs. Based on Tarski's hierarchy for truth.
Notation: $\mathsf{Belief}\langle n, \tau \rangle$.

\end{description}

\section{Epistemology}

\begin{description}[style=nextline,leftmargin=2cm]

\item[AGM]
The Alchourr\'{o}n-G\"{a}rdenfors-Makinson framework for belief revision,
defining postulates that revision operations should satisfy.

\item[Agrippa's Trilemma]
The epistemological problem that justification either (1) stops at
unjustified beliefs (dogmatism), (2) continues infinitely, or
(3) is circular. CLAIR accepts pragmatic dogmatism.

\item[Coherentism]
The epistemological view that beliefs are justified by their coherence
with other beliefs, not by foundations.

\item[Foundationalism]
The epistemological view that some beliefs are self-justifying
foundations that support other beliefs.

\item[Myth of the Given (Sellars)]
The critique that no beliefs are pre-conceptually self-justifying.
Applies to LLMs: all inputs are theory-laden by training.

\item[Pragmatic Dogmatism]
CLAIR's stance on Agrippa's trilemma: accepting foundations as
pragmatic stopping points (not self-evident truths), mitigated
by fallibilism and transparency.

\item[Reliabilism]
The epistemological view that justification depends on the reliability
of belief-forming processes, not access to reasons.

\item[Stratified Coherentism]
CLAIR's epistemological architecture: a coherentist structure with
stratified levels (Level 0: training, Level 1: basic, Level 2+: derived).

\end{description}

\section{Multi-Agent}

\begin{description}[style=nextline,leftmargin=2cm]

\item[Arrow's Impossibility Theorem]
The theorem that no aggregation rule satisfies all of: unrestricted
domain, Pareto efficiency, independence of irrelevant alternatives,
and non-dictatorship. CLAIR escapes by restricting domain.

\item[Framework Compatibility]
The requirement that agents share enough conceptual structure for
meaningful aggregation. CLAIR checks this before combining beliefs.

\item[Pragmatic Internal Realism]
CLAIR's position on truth: truth is objective within shared frameworks
but framework-relative. Based on Putnam's internal realism.

\item[Trust Profile]
A structure tracking how much an agent trusts another, with domain-specific
trust levels that evolve based on track record.

\end{description}

\section{Implementation}

\begin{description}[style=nextline,leftmargin=2cm]

\item[Hash-Consing]
A technique for sharing identical substructures by hashing their content.
CLAIR uses this for justification DAGs.

\item[Invalidation Condition]
A condition specifying when a belief should be reconsidered, such as
input changed, assumption false, confidence below threshold, or timeout.

\item[Reference Interpreter]
A minimal implementation prioritizing clarity and correctness over
performance. CLAIR's reference interpreter is specified in Haskell.

\item[Strict Evaluation]
Evaluating arguments before function application. CLAIR uses strict
evaluation to ensure confidence is computed at derivation time.

\end{description}

\section{Formal Verification}

\begin{description}[style=nextline,leftmargin=2cm]

\item[Lean 4]
A dependently-typed programming language and theorem prover. CLAIR's
confidence algebra is formalized in Lean 4 with Mathlib.

\item[Mathlib]
The mathematical library for Lean 4, providing formalized mathematics
including the unit interval used for confidence.

\item[unitInterval]
Mathlib's formalization of the closed interval $[0,1]$ in $\mathbb{R}$,
used as the basis for CLAIR's Confidence type.

\end{description}

\section{Phenomenology}

\begin{description}[style=nextline,leftmargin=2cm]

\item[Functional State]
A state characterized by its causal/computational role rather than
its intrinsic nature. CLAIR describes functional belief states.

\item[Hard Problem (Chalmers)]
The problem of explaining why there is subjective experience at all,
as opposed to just information processing.

\item[Heterophenomenology (Dennett)]
The method of studying consciousness by treating subjects' reports
as data to be explained, not privileged evidence of internal states.

\item[Phenomenal State]
A state with subjective experiential quality (``what it's like'').
Whether LLMs have phenomenal states is undetermined.

\end{description}

\section{Abbreviations}

\begin{description}[style=nextline,leftmargin=2cm]

\item[AGM] Alchourr\'{o}n-G\"{a}rdenfors-Makinson (belief revision theory)
\item[ATMS] Assumption-based Truth Maintenance System
\item[CLAIR] Comprehensible LLM AI Intermediate Representation
\item[CPL] Confidence-Bounded Provability Logic
\item[DAG] Directed Acyclic Graph
\item[DEL] Dynamic Epistemic Logic
\item[GL] G\"{o}del-L\"{o}b (provability logic)
\item[JTMS] Justification-based Truth Maintenance System
\item[LLM] Large Language Model
\item[SCC] Strongly Connected Component
\item[TMS] Truth Maintenance System

\end{description}
